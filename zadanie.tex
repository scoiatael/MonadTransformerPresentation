\documentclass[11pt,wide]{mwart}

\usepackage[OT4,plmath]{polski}
\usepackage{amsmath,amssymb,amsfonts,amsthm,mathtools}
\usepackage{color}
\usepackage{fontspec}
\usepackage{listings,times}

\definecolor{keywords}{RGB}{30,200,30}
\definecolor{comments}{RGB}{210,40,40}
\usepackage{listings,times}
\lstset{language=Haskell,
basicstyle=\ttfamily\small,
breaklines=true,
showstringspaces=false,
keywordstyle=\color{keywords},
commentstyle=\color{comments}\emph}

\usepackage{bbm}
\usepackage[colorlinks=true, urlcolor=blue]{hyperref}
\usepackage{url}


\title{Zadanie z transformatorów monad}
\author{Łukasz Czapliński}
\date{\today}

\begin{document}
\maketitle
\section{Do kiedy, komu?}
Zadanie należy wysłać do 21.11.2013 r. na \href{mailto:eternal.kadir@gmail.com}{\nolinkurl{eternal.kadir@gmail.com} }. Tam też należy kierować wszelkie pytania, prośby, podania i inne, związane z tematyką wykładu lub nie, sugestie. Oceniający zastrzega sobie prawo do zignorowania wszelkich otrzymanych dokumentów, za wyjątkiem zadań do oceny. 
\section{Co?}
\subsection{Wstęp teoretyczny}
Niech język termów będzie opisany jako \begin{lstlisting} % (fold)

data Term = And Term Term | Or Term Term 
  | Var Variable | Not Term 
  | Let Variable Term Term | Const Value
  | Appl Function Term deriving

\end{lstlisting}Wówczas możemy interpretować dany term poleceniem \begin{lstlisting} % (fold)

interp :: InterpMonad m => Term -> m Value

\end{lstlisting} gdzie InterpMonad jest na przykład następującą klasą: \begin{lstlisting} % (fold)

class (MonadState m, MonadEnv m) => 
  InterpMonad m where
    start :: (Show a, InterpMonad m) => m a -> String

  \end{lstlisting} której instancja może być pewnym złożeniem transformatorów monad. Wówczas każdy transformator będzie spełniał swoje zadanie, a jedynym problemem jest odpowiednie "liftowanie" operacji. Zostało to omówione na wykładzie,\begin{itemize}\itemsep2pt
  \item kod: \url{https://github.com/scoiatael/MonadTransformerExample},
  \item prezentacja: \url{https://github.com/scoiatael/MonadTransformerPresentation}.
\end{itemize}
\subsection{Do rzeczy}
Niech dany będzie język, w którym termy są opisane w następujący sposób:\begin{lstlisting} % (fold)

data Term = And Term Term | Or Term Term 
  | Var Variable | Not Term 
  | Let Variable Term Term | Const Value
  | Twofold Variable Term  deriving (Read, Show, Ord, Eq)
  \end{lstlisting}
  Natomiast interp będzie zwracał monadę będącą instancją klasy
  \begin{lstlisting}
class (MonadState m, MonadWriter m, MonadList m, MonadError m) => 
  InterpMonad m where
    start :: (Show a, InterpMonad m) => m a -> String

class Monad m => MonadWriter m where
  tell :: String -> m ()

\end{lstlisting}Zadaniem jest napisać interpreter tego języka. Interpreter ten powinien spełniać następujące kryteria:\begin{itemize}\itemsep2pt
  \item instrukcje let, twofold powinny zostawiać ślad w logu (zrealizowanym przez tell MonadWriter), aby możliwe było prześledzenie historii obliczeń,
  \item twofold powinno być interpretowane w dany sposób:\begin{lstlisting} % (fold)

  interp Twofold v t -> do
    mv <- getVar v
    if mv /= Nothing
      then err $ "Variable " ++ v ++ " already taken at this point."
      else return ()
    let v1 = branch v True t
    let v2 = branch v False t
    merge v1 v2

  \end{lstlisting}
  branch v X t oblicza wartość t przy zmiennej x równej X;
  innymi słowy: twofold powinno pozwalać na sprawdzenie wyników pewnego wyrażenia dla wszystkich możliwych wartości zmiennej.
  \item and, or powinny być leniwe (można to sprawdzić używając twofold, w którym jedna z gałęzi obliczeń wywołuje wyjątek),
  \item semantyka pozostałych wyrażen powinna być taka, jak w przykładzie z wykładu.
\end{itemize}
Można napisać program od zera lub użyć znajdujących się pod adresem:\\ \url{https://github.com/scoiatael/MonadTransformerExample} (w katalogu Task) plików przykładowych (znajduje się tam gotowa definicja funkcji interp, mogą pojawiać się tam dodatkowe wskazówki).\\
Dodatkową pomocą może być lektura \cite{mtransformer}.
\subsection{Ocena}
Ocenie podlega:\begin{itemize}\itemsep2pt
  \item poprawność interpretera,
  \item możliwość jego rozszerzenia,
  \item styl, w jakim został napisany kod.
\end{itemize}
\vspace{3pt}
Dodatkowe punkty można zyskać za \begin{itemize}\itemsep2pt
  \item dobrą argumentację wybranego złożenia transformatorów (jaki ma wpływ na zachowanie interpretera),
  \item przysłaniem oprócz gotowego interpretera, także jego wersji rozszerzonej o obsługę funkcji (podobnie jak w przykładzie),
  \item dobre przedstawienie możliwości swojego interpretera (przykładowe programy, problemy jakie można nim rozwiązać).
\end{itemize}
\bibliographystyle{plain}
\bibliography{books}
\end{document}
